\documentclass[11pt]{article}

\usepackage{../linear}

\begin{document}

\coverpage{4}

% hw problem 1 -----------------------------------------------------------------

\begin{exercise}{1}
    \problem{
        Compute $ \| A \| _1$, $ \| A \| _2$, and $ \| A \| _\infty$ where
        $$ A =
        \begin{bmatrix}
            3 & 1 & 1 \\
            0 & 2 & 0 \\
            1 & -1 & 3
        \end{bmatrix}
        $$
        For $ \| A \| _2 $ you may use the fact that the characteristic polynomial of $A^T A$ is
        $$ P(\lambda) = \det (A^T A - \lambda I) = (\lambda - 2)(\lambda - 8)(\lambda - 16) $$
        Use the fact that $ \| A \| _2 = \sqrt{\lambda _{max}} $ where $\lambda _{max}$ is the largest eigenvalue of $A^T A$.
    }
    \answer{
        First we compute $ \| A \| _1$:
        $$ \| A \| _1 = \max _j \sum _{i=1}^2 | a_{ij} | = \max \{ 4, 4, 4 \} = 4 $$
        Then we compute $ \| A \| _2$ (note that $\lambda_{max}$ is the largest eigenvalue of $A^T A$):
        $$ \| A \| _2 = \sqrt{\lambda_{max}} = \sqrt{16} = 4 $$
        Finally, we compute $ \| A \| _\infty$:
        $$ \| A \| _\infty = \max _i \sum _{j=1}^2 | a_{ij} | = \max \{ 5, 2, 5 \} = 5 $$
    }
\end{exercise}

\newpage

% hw problem 2 -----------------------------------------------------------------

\begin{exercise}{2}
    \problem{
        Recall that the condition number $\kappa (A)$ of a matrix $A$ is defined by
        $$ \kappa (A) = \| A \| \hspace{0.43em} \| A ^{-1} \| $$
        The main theoerm conclusion related to the condition number is that
        $$ \dfrac{\| x - \xhat \|}{\| x \|} \leq \kappa (A) \dfrac{\| b - \bhat\|}{\| b \|} $$
        where $x$ and $\xhat$ are solutions to $Ax = b$ and $A\xhat = \bhat$.
        When $\kappa (A)$ is large, the system is said to be ill-conditioned and even small relative errors in $b$ can result in large relative errors in the solution $x$.
        In this problem
        $$ A = \begin{bmatrix} 1 & 2 \\ 3 & 6.01 \end{bmatrix}
        \hspace{2em}
        b = \begin{pmatrix} 3 \\ 9.01 \end{pmatrix}
        \hspace{2em}
        \bhat = \begin{pmatrix} 3 \\ 9 \end{pmatrix}
        $$
        and all norms are the $\infty$-norm as in $\| A \| _\infty$.
        \begin{enumerate}[label=\alph*)]
            \item Compute $A^{-1}$ exactly.
            \item Compute $x$ and $\xhat$ exactly.
            \item Find the relative errors below. Do they differ by a lot?
            $$ e_x = \dfrac{\| x - \xhat \|}{\| x \|} \hspace{2em} e_b = \dfrac{\| b - \bhat \|}{\| b \|} $$
            \item Compute the condition number
        \end{enumerate}
        $$ \kappa (A) = \| A \| _\infty \| A^{-1} \| _\infty $$
    }
    \answer{
        \begin{enumerate}[label=\alph*)]
            \item We compute the exact inverse of $A$.
            Note that $\det A = 1/100$.
            Then the exact inverse of $A$ is
            $$A^{-1} = \dfrac{1}{100} \begin{bmatrix} 6.01 & -2 \\ -3 & 1 \end{bmatrix} = \begin{bmatrix} 601 & -200 \\ -300 & 100 \end{bmatrix} $$
            \item We now compute $x$ and $\xhat$ exactly.
            $$
            x = A^{-1} b =
            \begin{bmatrix}
                601 & -200 \\
                -300 & 100
            \end{bmatrix}
            \begin{pmatrix}
                3 \\ 9.01
            \end{pmatrix}
            =
            \begin{pmatrix}
                1803 - 1802 \\ -900 + 901
            \end{pmatrix}
            =
            \begin{pmatrix}
                1 \\ 1
            \end{pmatrix}
            $$ \\
            $$
            \xhat = A^{-1} \bhat =
            \begin{bmatrix}
                601 & -200 \\
                -300 & 100
            \end{bmatrix}
            \begin{pmatrix}
                3 \\ 9
            \end{pmatrix}
            =
            \begin{pmatrix}
                1803 - 1800 \\ -900 + 90
            \end{pmatrix}
            =
            \begin{pmatrix}
                3 \\ 0
            \end{pmatrix}
            $$
            \item We now compute and compare the relative errors $e_x$ and $e_b$.
            $$
            e_x = \dfrac{\| x - \xhat \|}{\| x \|} =
            \dfrac{\| \begin{pmatrix} 1 \\ 1 \end{pmatrix} - \begin{pmatrix} 3 \\ 0 \end{pmatrix} \| }{\| \begin{pmatrix} 1 \\ 1 \end{pmatrix} \|} =
            \dfrac{\| \begin{pmatrix} -2 \\ 1 \end{pmatrix} \| }{\| \begin{pmatrix} 1 \\ 1 \end{pmatrix} \|} =
            \dfrac{2}{1} = 2
            $$ \\
            $$
            e_b = \dfrac{\| b - \bhat \|}{\| b \|} =
            \dfrac{\| \begin{pmatrix} 3 \\ 9.01 \end{pmatrix} - \begin{pmatrix} 3 \\ 9 \end{pmatrix} \| }{\| \begin{pmatrix} 3 \\ 9.01 \end{pmatrix} \|} =
            \dfrac{\| \begin{pmatrix} 0 \\ 0.01 \end{pmatrix} \| }{\| \begin{pmatrix} 3 \\ 9.01 \end{pmatrix} \|} =
            \dfrac{0.01}{9.01} = \dfrac{1}{901}
            $$
            The relative errors differ by a large amount.
            The relative error in our input variable is $e_b = 1/901 \approx 0.11\% $ which is rather small.
            However, our relative output error is $e_x = 2 = 200\% $, which is much larger than our input error.
            This suggests that our system is highly sensitive to small changes in initial conditions.
            \item We now compute $\kappa (A) = \| A \| _\infty \| A^{-1} \| _\infty$.
            Note that $ \| A \| _\infty = \max \{ 3, 9.01 \} = 9.01 $ and $\| A^{-1} \| _\infty = \max \{ 801, 400 \} = 801 $.
            Then we can compute the condition number:
            $$ \kappa (A) = \| A \| _\infty \| A^{-1} \| _\infty = 9.01 * 801 = 7217.01 $$
        \end{enumerate}
    }
\end{exercise}

% hw problem 3 -----------------------------------------------------------------

\begin{exercise}{3}
    \problem{
        Recall the general iterative technique for solving $Ax = b$ has the split equation
        $$ Q x_{n+1} = (Q - A) x_n + b $$
        which may be written as
        $$ x_{n+1} = K x_n + c \hspace{2em} K = I - Q^{-1}A \hspace{2em} c = Q^{-1} b $$
        In all of the following questions, we have
        $$ A = \begin{bmatrix} 20 & 1 \\ -1/2 & 2 \end{bmatrix}
        \hspace{2em}
        b = \begin{pmatrix} 120 \\ 159 \end{pmatrix}
        \hspace{2em}
        x_0 = \begin{pmatrix} 0 \\ 0 \end{pmatrix}
        $$
        \begin{enumerate}[label=\alph*)]
            \item Use the matlab function ``Iterate.m'' to approximate the solution of $Ax = b$ using the Gauss-Seidel, Jacobi, and Richardson iteration techniques for $N = 10$ iterates each.
            For each method, print out the iteration matrix $x$ and state if the method converges.
            \item For the Gauss-Seidel technique, $\| K \| _1 = 1/16$.
            Use this and the fact that $\| e_{n+1} \| \leq \| K \| ^n \| e_0 \| = \| K \| ^n \| x \|$ (when $x_0 = \mathbf{0}$)  to find the minimum value for $n$ that ensures the following relative error tolerance:
        \end{enumerate}
        $$ \dfrac{\| x_{n+1} - x \|_1}{\| x \| _1} < 10^{-12} $$
    }
    \answer{
        \begin{enumerate}[label=\alph*)]
            \item The iteration matrix for Gauss-Seidel technique.
                This method converges.
                    \begin{center}
                    \begin{tabular}{|c|c|}
                        \hline
                        $x_1$             &   $x_2$             \\ \hline
                        0                 &   0                 \\ \hline
                        6.000000000000000 &  81.000000000000000 \\ \hline
                        1.950000000000000 &  79.987499999999997 \\ \hline
                        2.000625000000000 &  80.000156250000003 \\ \hline
                        1.999992187499999 &  79.999998046875007 \\ \hline
                        2.000000097656249 &  80.000000024414064 \\ \hline
                        1.999999998779296 &  79.999999999694822 \\ \hline
                        2.000000000015259 &  80.000000000003823 \\ \hline
                        1.999999999999809 &  79.999999999999957 \\ \hline
                        2.000000000000002 &  80.000000000000000 \\ \hline
                    \end{tabular}
                \end{center}
                The iteration matrix for Jacobi technique.
                This method converges.
                    \begin{center}
                    \begin{tabular}{|c|c|}
                        \hline
                        $x_1$             &   $x_2$             \\ \hline
                        0                 &   0                 \\ \hline
                        6.000000000000000 &  79.500000000000000 \\ \hline
                        2.025000000000000 &  81.000000000000000 \\ \hline
                        1.950000000000000 &  80.006249999999994 \\ \hline
                        1.999687500000000 &  79.987499999999997 \\ \hline
                        2.000625000000000 &  79.999921874999998 \\ \hline
                        2.000003906250000 &  80.000156250000003 \\ \hline
                        1.999992187499999 &  80.000000976562504 \\ \hline
                        1.999999951171874 &  79.999998046875007 \\ \hline
                        2.000000097656249 &  79.999999987792975 \\ \hline
                    \end{tabular}
                \end{center}
                The iteration matrix for Richardson technique (note that each value is multiplied by $10^{12}$).
                This method does not converge.
                    \begin{center}
                    \begin{tabular}{|c|c|}
                        \hline
                        $x_1$              &  $x_2$             \\ \hline
                        0                  &  0                 \\ \hline
                        0.000000000120000  &  0.000000000159000 \\ \hline
                        -0.000000002319000 &  0.000000000060000 \\ \hline
                        0.000000044121000  & -0.000000001060500 \\ \hline
                        -0.000000837118500 &  0.000000023280000 \\ \hline
                        0.000015882091500  & -0.000000441680250 \\ \hline
                        -0.000301317938250 &  0.000008382885000 \\ \hline
                        0.005716658061750  & -0.000159041695125 \\ \hline
                        -0.108457461358125 &  0.003017370885000 \\ \hline
                        2.057674395039375  & -0.057246101405062 \\ \hline
                    \end{tabular}
                \end{center}
            \item We now want to know the smallest $n$ that satisfies $ \dfrac{\| x_{n+1} - x \| _1}{\| x \| _1} < 10^{-12}$.
            Since we have chosen $x_0 = \mathbf{0}$, we can say that
            $$ \| x_{n+1} - x \| _1 = \| e_{n+1} \| _1 \leq \| K \| ^n_1 \| e_0 \| _1 = \| K \| ^n_1 \| x \| _1 $$
            But then we can simplify:
            $$ \dfrac{\| x_{n+1} - x \| _1}{\| x \| _1}  \leq \dfrac{\| K \| _1^n \| x \| _1}{\| x \| _1} = \| K \| _1 ^n $$
            So we need only wonder what is the smallest $n$ that satisfies $ \| K \| ^n_1 < 10^{-12} $:
            $$ \| K \| _1 ^n = 1/16^n < 10^{-12} = 1/10^{12} \iff 16^n > 10^{12} \iff n > \lfloor 12 \log _{16} (10) \rfloor $$
            But $\lfloor 12 \log _{16} (10) \rfloor = \lfloor 12 \dfrac{\log 10}{\log 16} \rfloor = \lfloor 12 / \log 16 \rfloor = 9$ so we must have $n > 9$.
            Therefore, the smallest $n$ that satisfies $ \dfrac{\| x_{n+1} - x \| _1}{\| x \| _1} < 10^{-12}$ is $n = 10$.
        \end{enumerate}
    }
\end{exercise}

\end{document}
