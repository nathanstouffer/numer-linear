\documentclass[11pt]{article}

\usepackage{../linear}

\begin{document}

\coverpage{1}

% hw problem 1 -----------------------------------------------------------------

\begin{exercise}{1}
    \problem{
        The real number $x = 11/8$ has decimal representation $x_d = 1.375$ and binary represenatation $x_b = 1.011$.
        Compute each of the following \underline{relative errors} in decimal.
        $$ E_d = \dfrac{|x_d - chop(x_d,3)|}{x_d} \hspace{2em} E_b = \dfrac{|x_b - chop(x_b,3)|}{x_b} $$
    }
    \answer{
        Let's begin with $E_d = \dfrac{|1.375 - chop(1.375,3)|}{1.375} = \dfrac{|1.375 - 1.37|}{1.375} = \dfrac{0.005}{1.375} = \dfrac{1}{275} = 0.0036 $. \parspace \parspace
        Seperately, we can compue $E_b$.
        In binary, the numerator is $|1.011 - chop(1.011,3)| = |1.011 - 1.01| = 0.001$.
        Translating to decimal, we get $1/8$ for the numerator.
        So $E_b$ is
        $$ E_b = \dfrac{1/8}{11/8} = \dfrac{1}{8} * \dfrac{8}{11} = \dfrac{1}{11} $$
    }
\end{exercise}

% hw problem 2 -----------------------------------------------------------------

\begin{exercise}{2}
    \problem{
        Suppose one can compute $\sqrt x$ exactly but an error of $\delta > 0$ is incurred by some finite representation $\xhat$ of $x$. \\\\
        \indent a) For $\delta > 0$ find a uniform upper bound on the absolute error $E_a = |\sqrt{x} - \sqrt{\xhat}|$ valid for all $x \in [0,1]$. \\\\
        \indent b) If $\delta = 10^{-6}$ what does a) imply the upper bound on $E_a$ is on $[0,1]$?
    }
    \answer{ \\\\
        \indent a) We begin by finding a uniform upper bound for the absolute error $E_a = |\sqrt x - \sqrt{\xhat} |$:
        $$ E_a = |\sqrt{x} - \sqrt{\xhat}| = |\sqrt{x} - \sqrt{\xhat}| * \dfrac{\sqrt{x} + \sqrt{\xhat}}{\sqrt{x} + \sqrt{\xhat}} = \dfrac{|(\sqrt{x} - \sqrt{\xhat}) * (\sqrt{x} + \sqrt{\xhat})|}{\sqrt{x} + \sqrt{\xhat}} = \dfrac{|x - \xhat|}{\sqrt{x} + \sqrt{\xhat}}$$
        \indent \hspace{1em} Now we substitute $\xhat = x + \delta$:
        $$ E_a = \dfrac{|x - x - \delta|}{\sqrt{x} + \sqrt{x + \delta}} = \dfrac{\delta}{\sqrt{x} + \sqrt{x + \delta}} $$
        \indent \hspace{1em} Since $x \in [0,1]$,
        $$ E_a \leq max \{ E_a \} = E(0, \delta) = E(\delta) = \dfrac{\delta}{\sqrt{0} + \sqrt{0 + \delta}} = \dfrac{\delta}{\sqrt{\delta}} = \sqrt{\delta} $$
        \indent \hspace{1em} Therefore, $E_a \leq E(\delta) = \sqrt{\delta} \hspace{0.5em} \forall x \in [0,1]$. \\\\
        \indent b) If $\delta = 10^{-6}$, then part a) implies that the upper bound for $E_a$ on $[0,1]$ is
        $$ \sqrt{\delta} = \sqrt{10^{-6}} = (10^{-6})^{1/2} = 10^{-6/2} = 10^{-3} $$
    }
\end{exercise}

% hw problem 3 -----------------------------------------------------------------

\begin{exercise}{3}
    \problem{
        The Taylor series for $f(x) = ln(1 + x)$ is
        $$ ln(1 + x) = \sum _{k=1}^n (-1)^{k-1} \dfrac{x^k}{k} + E_n (\zeta, x) = P_n(x) + E_n(\zeta, x) $$
        and converges for $ x \in (-1, 1] $. \\\\
        \indent a) Use the Alternating Series Test to bound the error $|E_n|$ by $\hat{E}_n$. Use $\hat{E}_n$ to find an $n$ sufficiently \indent \hspace{1em} large so that
        $$ |ln(2) - P_n(1)| \leq \hat E_n \leq 10^{-6} $$
        \indent \hspace{1em} Here $ x = 1 $. \\\\
        \indent b) One can accelerate the series convergence rate using the following identity
        $$ ln(2) = ln(e*2/e) = 1 + ln(2/e) = 1 + ln( 1 + (2/e - 1) ) = 1 + ln(1 + x) $$
    }
    \answer{ \\\\
        \indent a) We begin by finding a bound $\hat{E}_n$ for $|E_n|$.
        By the Alternating Series Theorem,
        $$ |E_n| = |f(x) - P_n(x)| \leq |a_{n+1}|$$
        \indent \hspace{1em} So we can compute the upper bound as
        $$ |E_n| = |(-1)^{n+1-1} \dfrac{x^{n+1}}{n+1}| = \dfrac{|x^{n+1}|}{n+1} \leq \hat{E}_n = \dfrac{1}{n+1} $$
        \indent \hspace{1em} For $| ln(2) - P_n(1)| \leq \hat{E}_n \leq 10^{-6}$, we must have $n \geq 999999$ so that
        $$ \hat{E}_n \leq 1/(999999 + 1) = 1/1000000 = 10^{-6} $$
        \indent b) Using the identity $ln(2) = 1 + ln(1 + x)$ where $x = 2/e - 1$, we can accelerate the convergence.
        \indent \hspace{1em} Consider the following table generated using MATLAB and this new formula.

        \begin{center}
            \begin{tabular}{|c|c|c|}
                \hline
                $n$ &  $1 + P_n(x)$       &  $E_a = | 1 + P_n(x) - ln(2) |$     \\ \hline
                1   &  0.735758882342885  &  0.042611701782939                  \\ \hline
                2   &  0.700847198212544  &  0.007700017652599                  \\ \hline
                3   &  0.694697129923282  &  0.001549949363336                  \\ \hline
                4   &  0.693478304234465  &  0.000331123674520                  \\ \hline
                5   &  0.693220653144671  &  0.000073472584726                  \\ \hline
                6   &  0.693163918134727  &  0.000016737574782                  \\ \hline
                7   &  0.693151068086924  &  0.000003887526978                  \\ \hline
                8   &  0.693148097014804  &  0.000000916454859                  \\ \hline
                9   &  0.693147399166433  &  0.000000218606488                  \\ \hline
                10  &  0.693147233206223  &  0.000000052646278                  \\ \hline
            \end{tabular}
        \end{center}
    }
\end{exercise}

\end{document}
