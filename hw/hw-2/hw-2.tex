\documentclass[11pt]{article}

\usepackage{../linear}

\begin{document}

\coverpage{2}

% hw problem 1 -----------------------------------------------------------------

\begin{exercise}{1}
    \problem{
        Use Theorem 2.4 of the notes to prove
        $$ h - sin(h) = O(h^3) \hspace{2em} as \; h \to 0 $$
        Theorem 2.4 states that for functions $f(h)$ and $g(h)$ that are continuous near $h = 0$ and
        $$lim_{h \to 0} \dfrac{f(h)}{g(h)} = L < \infty $$
        then $f = O(g)$ as $ h \to 0 $.
    }
    \answer{
        To show that $h - sin(h) = O(h^3)$, we will show that $lim_{h \to 0} f(h) / g(h) = L < \infty$ for some $L \in \R$.
        $$ lim_{h \to 0} \dfrac{h - sin(h)}{h^3} = \dfrac{0 - sin(0)}{0^3} = \dfrac{0}{0} $$
        Since the limit is $0/0$, we can apply L'Hospital's rule to say that
        $$ lim_{h \to 0} \dfrac{h - sin(h)}{h^3} = \dfrac{1 - cos(h)}{3h^2} = \dfrac{1 - cos(0)}{3*0^2} = \dfrac{1-1}{0} = \dfrac{0}{0} $$
        By the same reasoning, we apply L'Hospital's rule again
        $$ lim_{h \to 0} \dfrac{h - sin(h)}{h^3} = \dfrac{sin(h)}{6h} = \dfrac{sin(0)}{6*0} = \dfrac{0}{0} $$
        We then apply L'Hospital's rule a final time to say that
        $$ lim_{h \to 0} \dfrac{h - sin(h)}{h^3} = \dfrac{cos(h)}{6} = \dfrac{cos(0)}{6} = \dfrac{1}{6} $$
        Since $lim_{h \to 0} \dfrac{h - sin(h)}{h^3} = 1/6 < \infty$, we say that $h - sin(h) = O(h^3)$.
    }
\end{exercise}

% hw problem 2 -----------------------------------------------------------------

\begin{exercise}{2}
    \problem{
        Using Taylor series of $f(x+h)$ and $f(x-h)$ show
        $$ f(x+h) - 2f(x) + f(x-h) = f''(x)h^2 + O(h^4) $$
        Then, solve for $f''(x)$ to get an approximation for $f''(x)$ and state the order of the truncation error.
    }
    \answer{
        We know the Taylor series expansion of $f(x+h)$ and $f(x-h)$ to be
        $$ f(x+h) = f(x) + f'(x)h + \dfrac{1}{2!}f''(x)h^2 + \dfrac{1}{3!}f^{(3)}(x)h^3 + O(h^4) $$
        $$ f(x-h) = f(x) - f'(x)h + \dfrac{1}{2!}f''(x)h^2 - \dfrac{1}{3!}f^{(3)}(x)h^3 + O(h^4) $$
        Adding these equations together, we get
        $$ f(x+h) + f(x-h) = 2f(x) + 0 + f''(x)h^2 + 0 + O(h^4) $$
        which is equivalent to saying that
        $$ f(x+h) - 2f(x) + f(x-h) = f''(x)h^2 + O(h^2) $$
        which we were required to show.
        From here, we can rearrange the above equation to say that
        $$ f''(x) = \dfrac{f(x+h) - 2f(x) + f(x-h)}{h^2} + \dfrac{O(h^4)}{h^2} = \dfrac{f(x+h) - 2f(x) + f(x-h)}{h^2} + O(h^2) $$
        Thus we can approximate $ f''(x) \approx \dfrac{f(x+h) - 2f(x) + f(x-h)}{h^2} $ with $O(h^2)$ error.
    }
\end{exercise}

% hw problem 3 -----------------------------------------------------------------

\begin{exercise}{3}
    \problem{
        Let $f(x) = (x-1) (x-3) (x-4)$.
        What root of $f(x)$ does the Bisection Method converge to on the interval $[a,b] = [0,5]$.
        Sketch $f(x)$ and label the first two midpoint values $x_1, x_2$.
    }
    \answer{
        Based on the graph below, we can say that the Bisection Method converges to $x^* = 0$ on the interval $[a,b] = [0,5]$.
        We now display the graph of $f(x) = (x-1) (x-3) (x-4)$ and label the first two midpoints $x_1$ and $x_2$.
        \newcommand{\range}{6}
        \begin{center}
            \begin{tikzpicture}[scale=1]
                % axes
                \draw[->] (-\range, 0) -- (\range, 0) node[right] {$x$};
                \draw[->] (0, -\range) -- (0, \range) node[above] {$y$};
                % labels for actual zeros
                \draw[black,fill=black] (1,0) circle (.25ex) node[align=left, below] {1 \hspace{1em}};
                \draw[black,fill=black] (3,0) circle (.25ex) node[align=left, below] {3 \hspace{1em}};
                \draw[black,fill=black] (4,0) circle (.25ex) node[align=right, below] {\hspace{1em} 4};
                % function
                \draw[scale=1, domain=0.3:4.75, smooth, variable=\x, black] plot ({\x}, {(\x-1)*(\x-3)*(\x-4)}) node[align=right, above] {$f(x)$};
                % labels for bisection method
                \draw[blue,fill=blue] (0,0) circle (.25ex) node[align=left, above] {$a$ \hspace{1em} };
                \draw[blue,fill=blue] (5,0) circle (.25ex) node[align=left, above] {$b$ \hspace{1em} };
                \draw[blue,fill=blue] (2.5,0) circle (.25ex) node[align=left, above] {$x_1$ \hspace{1em} };
                \draw[blue,fill=blue] (1.25,0) circle (.25ex) node[align=right, below] {\hspace{1em} $x_2$};
            \end{tikzpicture}
        \end{center}
    }
\end{exercise}

% hw problem 4 -----------------------------------------------------------------

\begin{exercise}{4}
    \problem{
        For the following three problems use modified versions of the posted $Newton.m$, $f.m$, and $df.m$ matlab files to find an approximation of a root of $f(x)$ using the indicated starting guess $x_1$.
        For each case, print output in the three columns
        $$ n \hspace{2em} x_n \hspace{2em} E_n \hspace{4em} (1 \leq n \leq 10) $$
        Lastly, state if the convergence rate is linear, superlinear, or quadratic. \\\\
        \indent a) Newton's Method: $x_1 = 4, f(x) = (x-1)^2 - 2$ \\\\
        \indent b) Newton's Method: $x_1 = 4, f(x) = (x-1)^2$ \\\\
        \indent c) Accelerated Newton's Method: $x_1 = 4, f(x) = (x-1)^2$ % \lambda = 2?
    }
    \answer{

    }
\end{exercise}

% hw problem 5 -----------------------------------------------------------------

\begin{exercise}{5}
    \problem{
        Steffensen method for solving $f(x) = 0$ is defined by:
        $$ x_{n+1} = x_n - \dfrac{f(x_n)}{g(x_n)} = x_n - F(x_n) $$
        where
        $$ g(x) = \dfrac{f(x + f(x)) - f(x)}{f(x)} $$
        For simple roots $\xbar$ where $f(\xbar) \neq 0$, one can show the method has quadratic convergence.
        Write a Matlab file Steffensen.m whose output is
        $$ n \hspace{2em} x_n \hspace{2em} E_n $$
        for $1 \leq n \leq 10$ for the case $f(x) = x^2 - 4$ and $x_0 = 1.5$. Include your code for Steffensen.m and output.
    }
    \answer{

    }
\end{exercise}

\end{document}
